\section{Zero-Padding}
	Sea la señal en tiempo discreto:
	\begin{equation}
				x[n] = \begin{cases}
							cos\left( \frac{\pi n}{20} \right)  & \qquad : 0 \leq n \leq 99 \\
							0  & \qquad : \text{E.O.C.}
						\end{cases}
	\end{equation}
	
	A partir de la señal, se calcula la DFT para $N = 100$ y $N = 400$, graficando la magnitud del resultado en el intervalo $\left[ -\pi , \pi \right]$:
	\begin{figure}[H]
		\center
		\includegraphics[width=0.6\textwidth,clip, trim = {1.9cm 6.8cm 2.3cm 7cm}]{../plot_out/3_1.pdf}
		\caption{Resultado de la magnitud, para la DFT de la señal para N = 100 y N = 400. }
		\label{fig:3_1_mag_plot}
	\end{figure}
				